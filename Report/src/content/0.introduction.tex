\section{Introduction}

Our project consists in developing a ball stabilization model using a 3 degree of freedom (3dof) Steward platform. In previous years, other student groups have developed the physical model. This year, we have two main objectives: to stabilize the ball on the model and to make our model usable for the automatic laboratory at the Institut National des Sciences Appliquées (INSA) of Toulouse.
In this report, we focus on exploring the regulation of ball position on a 3dof Stewart platform using a Proportional Integral Derivative (PID) control system.
\newline\newline
Previous research has primarily concentrated on either 2dof \cite{kumar_design_2016, adiprasetya_implementation_2016} or 6dof \cite{lee_position_2003, koszewnik_pid_2014} Stewart platforms, with limited investigations conducted on the 3dof variant. Addressing this knowledge gap, our research aims to assess the feasibility of employing a PID controller to achieve accurate ball position regulation on a 3dof Stewart platform.
\newline\newline
Firstly, we will present the hardware employed and explore the integration of the Arduino microcontroller in the system, explaining its role and functionality in facilitating real-time control and communication with the platform.
\newline
Secondly, we will discuss the development and implementation of the ball detection algorithm, highlighting its effectiveness and reliability in accurately determining the position of the ball.
\newline
Furthermore, the mathematical model of the 3dof Stewart platform will be presented, elucidating the relationships and equations utilized for control and manipulation.
\newline
The report will finally examine the theoretical foundations and principles underlying the PID control system, exploring its potential capabilities and limitations in regulating the ball position. Although we couldn't perform practical experiments in this regard, we will present a comprehensive theoretical framework for understanding the PID control system's functioning and its potential application to the 3dof Stewart platform.
\newline\newline
While practical experimentation with the PID control system was limited, this report aims to provide a comprehensive understanding of the various components involved in the regulation of ball position on the 3dof Stewart platform. By focusing on the theoretical calculations, algorithm development, platform modeling, and the integration of the Arduino, this report lays the foundation for future practical implementation and testing.




